\documentclass[a4paper,12pt]{report}
\usepackage[utf8x]{inputenc}
\usepackage[T1]{fontenc}
\usepackage[french]{babel} 
\usepackage{lmodern}
\usepackage{graphicx}
\usepackage{svg}
\usepackage{makeidx}

\graphicspath{{img/}}

\title{Industries culturelles et politisation de l'esthétique}
\author{\textsc{Awada} Ali, \textsc{Baasandorj} Chinbat et \textsc{Jorandon} Guillaume}
\date{PH03 - P18\\ Université de technologie de Compiègne\\\vspace{1cm}\includesvg[width=5cm]{logo_utc.svg}}

\makeindex
\begin{document}

\maketitle

\tableofcontents

\chapter*{Introduction}

La réf~\cite{wow_pandemic}.

\chapter{Qu'est-ce qu'une industrie culturelle ?}

\section{Histoire des industries culturelles}

\section{Les industries culturelles à l'ère du numérique}

\chapter{Plateformes de diffusion de masse et algorithmes}

\section{Comment les algorithmes de profilage sélectionnent le contenu}

\section{Conséquences}

\subsection{Uniformisation des modes de pensée}

\subsection{Politisation de l'esthétique}

\subsection{Discriminations algorithmiques}

\subsection{Enjeux de pouvoir et de gouvernance}

\chapter{Vers une autre culture de l'information}

\section{Une autre approche du décisionnel}

\section{Vers la néguentropie}

\chapter*{Conclusion}

\bibliographystyle{alpha}
\bibliography{memoire_ph03_bib}

\printindex
\end{document}

