\documentclass[a4paper,12pt]{report}
\usepackage[top=2.5cm, bottom=2.5cm, left=2cm, right=2cm]{geometry}
\usepackage[utf8]{inputenc}
\usepackage[T1]{fontenc}
\usepackage[french]{babel} 
\usepackage{lmodern}
\usepackage{graphicx}
\usepackage{svg}
\usepackage{makeidx}
\usepackage[pdfusetitle]{hyperref}
\hypersetup{hidelinks,backref=true,pagebackref=true,hyperindex=true,colorlinks=false,breaklinks=true,urlcolor= custom_color,bookmarks=true,bookmarksopen=false}
\usepackage[acronyms, toc]{glossaries}
\makeglossaries
\usepackage{xparse}
\DeclareDocumentCommand{\newdualentry}{ O{} O{} m m m m } {
\longnewglossaryentry{gls-#3}{name={#5},text={#5},#1
}{#6}
\makeglossaries
\newacronym[see={[Voir :]{gls-#3}},#2]{#3}{#4}{#5\glsadd{gls-#3}}
}


\graphicspath{{img/}}
\longnewglossaryentry{francfort}{name={école de Francfort}, sort={ecole de francfort}}{Courant philosophique né dans les années 1950, empruntant notamment au marxisme (\textsc{Marx}) et au structuralisme (\textsc{Saussure}, \textsc{Barthes}, \textsc{Lévi-Strauss}). Il compte parmi ses membres notamment \textsc{Horkeimer} (sociologue allemand) ou \textsc{Adorno}.}
\longnewglossaryentry{big-data}{name={big data}}{Ce terme définit plusieurs aspects. Il s'agit d'abord d'un mot pour désigner les très gros volumes de données (des volumes tellement importants qu'ils ne peuvent être analysés par des humains), mais aussi leur stockage (\textit{data warehouse}) et leur traitement (\textit{data mining}). Ces traitements sont complètement automatisés, leur volume excluant toute possibilité de traitement manuel.}
\newacronym{gafam}{GAFAM}{Google, Apple, Facebook, Amazon, Microsoft} % faire une dual entry
\newacronym{acp}{ACP}{analyse en composantes principales}


\title{Industries culturelles et politisation de l'esthétique}
\author{\textsc{Awada} Ali, \textsc{Baasandorj} Chinbat et \textsc{Jorandon} Guillaume}
\date{PH03 - P18\\ Université de technologie de Compiègne\\\vspace{1cm}\includesvg[width=5cm]{logo_utc.svg}}

\makeindex
\begin{document}

\maketitle

\tableofcontents

\chapter*{Introduction}

Le 20\up{e} siècle a été une époque charnière dans la courte Histoire de l'humanité. Siècle des grandes guerres mondiales, il a été le témoin de projets techniques et scientifiques fulgurants, grandement motivés par les conflits et la lutte pour la suprématie. Dans l'après-guerre, on assiste progressivement à une globalisation et une industrialisation de la culture, phénomène étudié notamment par les philosophes de l'\gls{francfort}, comme Theodor \textsc{Adorno}. Ils développent ainsi dans les années 50 le concept d'industrie culturelle, et cherchent à étudier le fort pouvoir performatif de ces industries, étroitement liées à des enjeux économiques productivistes. \textsc{Adorno} postule notamment que les industries culturelles sont un des outils d'un pouvoir autoritaire et totalitaire moderne :

\begin{itemize}
    \item{autoritaire, car il impose une mentalité unique aux individus ;}
    \item{totalitaire, car s'exerce à grande échelle et à tous les niveaux ;}
    \item{moderne, car il n'a plus la forme des "dictatures d'hier".}
\end{itemize}

Il est important d'insister sur ce dernier point : les industries culturelles sont en effet devenues un instrument de pouvoir notamment à l'oeuvre dans nos belles démocraties du monde libre, au sein des sociétés libérales au sein desquelles la consommation de masse et le productivisme sont profondément ancrés.

Les conséquences sont multiples, et nous essaierons d'y revenir en détails dans ce mémoire. Nous nous attacherons en effet à comprendre comment s'articulent ces industries culturelles, au travers d'un exemple récent : les plateformes de diffusion de contenu en ligne. En effet, le 21\up{e} siècle fait la part belle, avec l'essor de l'informatique moderne, à la collecte et au traitement automatisé\footnote{Ce traitement est automatisé dans le sens où les volumes de données sont traités par des algorithmes qui synthétisent l'information pour lui donner un sens. Nous reviendrons dans ce rapport sur la nature de ce sens.} de grands volumes de données, que l'on désigne souvent sous le terme très marketing de \gls{big-data}. L'accès à un Internet haut débit pour une part croissante de la population mondiale nous permet de consommer toujours plus de produits culturels : musiques, films, séries, émissions de télévision, livres, vidéos YouTube... YouTube justement, est l'une des plateformes qui règne en maître sur cette ère connectée, aux côtés d'autres géants culturels comme Spotify, Amazon ou Netflix, plateforme sur laquelle nous reviendrons plus en détail. Ces plateformes utilisent massivent le \gls{big-data} et l'analyse de données pour différentes raisons : proposer de la publicité ciblée, catégoriser le contenu selon les utilisateurs, et les utilisateurs selon le contenu, extraire des profils types, parfois même faire de la science\footnote{Des chercheurs se sont montrés intéressés par la masse d'utilisateurs de certaines plateformes en ligne, qui peuvent être un terrain d'expérimentation intéressant. On citera par exemple le cas de "corrupted blood", un bug de World of Warcraft, célèbre MMORPG de l'éditeur Blizzard. Ce bug provoqua en 2005 une pandémie virtuelle, la diffusion massive d'une maladie affectant les avatars des joueurs. Des articles ont été publiés par des chercheurs enthousiastes, comme~\cite{wow_pandemic}.}...

On peut alors conjecturer que cette masse d'informations sur la population mondiale\footnote{Facebook par exemple revendique dans ses utilisateurs près d'un tiers de la population mondiale.}, dans les mains de ces quelques acteurs, leur confère un pouvoir politique et économique gigantesque. Elles s'inscrivent dans une logique de biopouvoir au sens de Michel \textsc{Foucault}, c'est à dire l'exercice du pouvoir sur la vie et les corps des individus. Nous nous sommes alors demandé comment ces nouveaux acteurs, avec leurs algorithmes de profilage, peuvent-ils façonner une esthétique politisée ? Comment construisent-ils les goûts et les choix du consommateur, quelle influence ont-ils sur notre perception de l'art, et du "beau" ? En quoi prennent-elles place dans des enjeux de pouvoir ? En quoi ces plateformes sont-elles politiques ? Pour répondre à cette question, nous reviendrons d'abord en détail sur la notion d'industrie culturelle, leur histoire, l'histoire de leur étude, et ses évolutions récentes. Puis nous étudierons les plateformes de diffusion de masse et leur impact : quels problèmes soulèvent-elles,  Enfin, nous essayerons de réfléchir à des alternatives pour essayer d'aborder différemment notre rapport à l'information et la culture.

Les sources utilisées pour rédiger ce mémoire sont indiquées à la fin du document, dans la section Bibliographie.

\chapter{Qu'est-ce qu'une industrie culturelle ?}

Pour commencer, il est essentiel de revenir sur la notion d'industrie culturelle, sur ses origines et ses évolutions récentes. Les industries culturelles sont un maillon important de notre monde industrialisé moderne, comme vecteurs de transmission des idées dominantes.

\section{Histoire des industries culturelles}

\subsection{Aux origines de la culture}

L'Homme s'est distingué dès le néolithique par sa capacité à extérioriser du savoir au travers d'objets techniques. Cette exosomatisation de la mémoire lui a permis de vivre et de survivre en s'adaptant à des environnements divers et variés, et a donc été un critère essentiel de différenciation (et de conflit) entre les peuples et les individus : c'est la naissance de la culture. Cette évolution ne s'est pas faite sur la génétique, puisqu'Homo Sapiens a relativement peu évolué depuis Néanderthal, mais sur ce que Bernard \textsc{Stiegler} appelle l'épiphylogénétique : il s'agit de facteurs d'évolution non héréditaires au sens génétique du terme, c'est à dire qui ne se transmettent pas par la reproduction. Ces caractères traversent l'Histoire grâce à des outils techniques, traces mémorielles du savoir du passé qui permettent de poursuivre la vie au delà du trépas individuel. Ces premières traces du passé, silex taillés, peintures rupestres, on les retrouve d'ailleurs aujourd'hui dans les musées d'Histoire de l'Homme. Ces traces se matérialisent au travers de ce que l'on peut appeler des rétentions tertiaires. Les rétentions sont un terme d'\textsc{Husserl}, et désignent ce qui est recueilli par la conscience. Husserl en distinguait trois types : rétentions immédiates, ou primaires ; et secondaires, à savoir les rétentions passées devenues des souvenirs. Mais \textsc{Stiegler} en introduit une troisième la rétention tertiaire, qui est une accumulation intergénérationnelle des mémoires individuelles, matérialisées donc au travers des outils techniques. On comprend alors la relation entre savoir et outil technique. Ce processus d'hominisation a permis l'émergence de sociétés et de cultures complexes, et les grecs ont bien résumé cette particularité de l'Homme au travers du mythe de Prométhée. Dans celui-ci, Prométhée, après avoir créé le vivant, a donné toutes les qualités au animaux, ce qui laisse l'Homme incomplet, car dépourvu d'une spécificité. Il se voit alors obligé de voler le feu (la technique) à Héphaïstos pour le donner aux humains. 


Nous parlions à l'instant de l'émergence des sociétés dans l'histoire de l'Homme. L'agriculture et l'écriture ont joué des rôles centraux dans ce processus. L'agriculture a concentré des individus autour de bassins fertiles, comme le Nil, et l'écriture dote notre espèce d'outils concrets dédiés à la conservation de la mémoire : des outils mnémotechniques. L'écriture a pu ainsi stabiliser les civilisations naissantes : par exemple, sur le Nil, les égyptiens ont pu la mettre à profit pour observer les étoiles, calculer et anticiper les crues du fleuve, et ainsi concevoir un calendrier adapté qui organise la vie des individus. Cette écriture est d'abord iconique (idéogrammes), mais les signes qui la composent se détachent peu à peu de leur relation au signifié pour devenir plus symboliques. L'apparition des systèmes alphabétiques permet une transmission de la pensée avec une exactitude que ne permettent pas les systèmes idéogrammatiques. Et c'est cette invention technique qui va permettre l'essor de la culture et du savoir sur les millénaires suivant qui constituent l'Histoire humaine. Dans L'origine de la géométrie, \textsc{Husserl} affirme même que l'existence de la géométrie serait rigoureusement impossible sans système d'écriture alphabétique, et on peut le croire sans trop de peine. Les systèmes alphabétiques conservent intacts les processus de pensée, ils sont à même de retranscrire exactement un discours, ce que ne peuvent pas faire par exemple les hiéroglyphes égyptiens.

Nous pouvons avancer jusq'au 15\up{e} siècle, avec l'arrivée de l'imprimerie qui modifie profondément notre rapport à la diffusion du savoir. Auparavant réservé à une élite érudie, notamment de moines copistes, l'imprimerie permet dans une certaine mesure une plus large diffusion des livres, qui sont déjà depuis Alexandrie un mode privilégié de transmission du savoir, tout du moins dans nos sociétés occidentales. Cette invention préfigure d'ailleurs l'essor des médias modernes, écrits d'abord (presse), puis hertziens (télévision, radio) et numériques (Internet). Dans une temporalité plus courte, elle est aussi en partie à l'origine des écrits de Martin \textsc{Luther} au début du 16\up{e} siècle, qui dénonça l'Église chrétienne pour ses vices et sa manipulation des masses, que \textsc{Luther} jugeait en décalage avec le message affiché\footnote{Il condamnera notamment fermement le commerce des indulgences, qui permettaient aux plus aisés d'acheter le pardon divin auprès de l'Église.}. Cette scission donna naissance au protestantisme et aux guerres de religion qui ont suivi, et il est intéressant de constater qu'elle n'aurait sans doute jamais eu lieu sans l'imprimerie, qui a permis de répandre la pensée luthérienne en Europe. Nous pouvons d'ailleurs revenir sur le mythe de Prométhée, sur sa fin que nous n'avons volontairement pas évoqué plus haut : en effet, Zeus a puni les Hommes\footnote{Prométhée fût lui aussi puni par Zeus, qui l'enchaîna au Caucase, condamné à avoir éternellement le foie dévoré le jour par un vautour, et réparé la nuit pour être à nouveau consommé le lendemain. Les dieux ont un sens du châtiment particulièrement vicieux.} en amenant la discorde en leur sein. En effet, incapables de s'entendre sur l'utilisation de la technique, de ces prothèses exosomatiques, les Hommes entrèrent alors en guerre. Et en effet, l'unité des groupes humains est sans cesse remise en cause par les évolutions de la technique : la technique dépasse l'Homme, et l'Homme dépasse la technique en permanence. D'où l'importance d'aborder les questions de justice et de sagesse dans leur utilisation, mais nous y reviendrons.

(mention du XIXe siècle et de l'entrée dans l'industrialisation, capitalisme, Germinal)

\subsection{Émergence des industries culturelles}

(partie de Chinbat à recopier) 

(faire intervention sur Packard pendant la partie de Chinbat)

\subsection{La question de la reproductibilité technique}

(introduction du livre + partie de Ali)

\section{Les industries culturelles à l'ère du numérique}

Nous l'avons vu, le 20\up{e} siècle a été une époque charnière pour l'essor des industries culturelles modernes. Mais à partir des années 60-70, une nouvelle révolution prépare le 21\up{e} siècle : l'informatique. Si \textsc{Babbage} pose dès le 19\up{e} siècle les bases de la machine à calculer, c'est pendant et après la guerre que tout s'accélère. Citons pêle-mêle John \textsc{Von Neumann} qui définit l'architecture de calculateur qui porte son nom, et qui à la base de toutes les architectures informatiques qui suivirent, et Alan \textsc{Turing}, qui avec son calculateur électromécanique\footnote{Calculateur à qui \textsc{Von Neumann} attribue d'ailleurs la véritable parentalité de l'architecture de \textsc{Von Neumann}.}, parvint à décrypter les messages chiffrés d'Énigma, système de chiffrement allemand, faisant prendre à la guerre un tournant décisif.

Avec l'abandon des parties mécaniques dans les calculateurs, ainsi que les progrès croissants sur la miniaturisation des transistors, leur puissance augmente de manière exponentielle, ce qui a conduit à la fameuse Loi de Moore ; en fait une conjecture, elle stipule que la puissance des ordinateurs double tous les deux ans. Jusqu'à maintenant, cette loi emprique s'est avérée, et elle a accompagné la naissance d'Internet, d'abord comme un projet militaire (ARPA), jusqu'aux foyers des particuliers à partir des années 1990.

Cette "révolution numérique"\footnote{Les anglophones parleront volontiers de \textit{digital revolution}, car le terme \textit{digital} en anglais renvoie au numérique (\textit{digits} : les chiffres). Il conviendra de ne surtout pas traduire cela par "révolution digitale", révolution qui ne concerne que l'utilisation que l'on a de nos doigts, et dont nous laissons le soin de l'interprétation au lecteur.} a permis l'émergence de nouveaux modes de communication, qui auraient encore été considérés comme de la science-fiction 50 ans avant : e-mails, forums, blogs, la génèse d'Internet s'est d'abord caractérisée par une large décentralisation de l'information, tant dans ses modes d'émission que de transmission. Mais plus récemment, Internet a été le terrain d'émergence de nouvelles gigantesques industries culturelles, que l'on tend à désigner en France sous l'acronyme \gls{gafam}, les 5 entreprises majeures de l'Internet "contemporain."

Il est légitime de se demander en quoi Facebook ou Apple peuvent être considérés comme des industries culturelles. Facebook n'est-il pas un réseau social, et Apple un fabricant d'ordinateurs\footnote{Apple vend aussi des smartphones, mais il est inutile de le mentionner : les smartphones sont des ordinateurs. Des ordinateurs qui se font passer pour des smartphones, mais des ordinateurs quand même. Cette stratégie qui consiste à cacher à l'utilisateur la nature réelle de ce qu'ils ont dans leur poche au quotidien est cruciale dans la stratégie globale d'Apple et des autres fabricants de smartphones, mais nous ne nous étenderons pas plus sur ce sujet. Cependant, la lecture de \cite{kids_computer} apporte quelques points d'analyse très intéressants.} ? Il se trouve que ces entreprises, malgré des secteurs d'activité différents, partagent plusieurs points communs.

Tout d'abord, ce sont des sociétés américaines, pays qui, comme nous l'avons vu plus tôt, a su imposer une hégémonie culturelle dans son histoire récente. Elles ont chacune, à un certain niveau, une manière de diffuser une certaine culture au sens large du terme. Il s'agit d'une culture avant tout consumériste et individualiste, sur un mode de pensée néo-libéral classique. Par exemple, Apple diffuse au travers de ses produits et de sa culture d'entreprise une certaine idée de l'humain, émancipé par la toute puissante technologie numérique, et cet idéal se communique notamment au travers de ses publicités. De plus, elles réunissent sous leur coupe une écrasante majorité du marché, chacune dans leur propre domaine. Facebook par exemple est le premier réseau social mondial, accueillant sur ses serveurs près du tiers de la population mondiale. Enfin, elles font un usage massif des techniques de \gls{big-data} afin d'effectuer un profilage de ses utilisateurs. Les raisons de ce profilage sont diverses, mais la principale est que les plateformes que proposent les \gls{gafam} reposent sur un modèle économique de publicité ciblé, où les utilisateurs paient de leurs données personnelles l'utilisation des plateformes, qui revendent ces données à des annonceurs publicitaires. Cela permet de proposer une publicité ciblée, et potentiellement encore plus efficace que la publicité du temps de Vance \textsc{Packard} dont nous avons parlé précédemment.

Nous allons avoir l'occasion de revenir plus en détail sur le fonctionnement et les conséquences de telles pratiques dans la suite de ce rapport, qui étudiera en détail un autre type d'industrie culturelle, plus évident : les plateformes de diffusion de contenu culturel.

\chapter{Plateformes de diffusion de masse et algorithmes}

Nous allons maintenant nous attacher à comprendre comment les plateformes de diffusion culturelle numériques fonctionnent, notamment dans leur manière de proposer le contenu à ses utilisateurs. Afin de définir clairement les termes de l'étude, arrêtons nous un instant sur ce que nous entendons par plateformes de diffusion culturelle numériqus, ou plateformes de diffusion de masse : il s'agit de portails en ligne qui mettent à disposition de leurs utilisateurs des produits culturels (livres, musique, films, séries, etc.) selon un modèle de paiement par produit (Amazon, iTunes store, services de VOD d'Orange ou SFR par exemple), ou par abonnement (Netflix, Spotify, Audible\footnote{Audible est une plateforme qui met à disposition des livres audio.}, Youtube Red...).

\section{Comment les algorithmes de profilage sélectionnent le contenu}

\section{Conséquences}

\subsection{Uniformisation des modes de pensée}

(Chinbat)

\subsection{Politisation de l'esthétique}

(ne pas en parler pour l'exposé, Chinbat le présente par l'exemple dans sa partie)

\subsection{Discriminations algorithmiques}

(Synthèse de ce qui est dit en SC22)

\subsection{Enjeux de pouvoir et de gouvernance}

(Pareil)

\section{Étude de cas : Netflix}

(Chinbat)

\chapter{Vers une autre culture de l'information}

\section{Une autre approche du décisionnel}

\section{Vers la néguentropie}

\chapter*{Conclusion}

\bibliographystyle{alpha}
\bibliography{memoire_ph03_bib} % Ajouter Vance packard, et autres références

%\printindex

\glossarystyle{altlist}
\newpage
\printglossary
\newpage
\printglossary[type=\acronymtype]

\end{document}

